\documentclass{beamer}
\usepackage[latin9]{inputenc}
\usepackage[french]{babel}
\usepackage{color} 
\usepackage{graphicx} 
\usepackage{listings}
\usepackage{amsmath}
\usepackage{amssymb}
\usepackage{mathptmx}
\usepackage{ifthen}

%Configuration du slide pour compiler la bonne version en fonction du besoin
%Part to configure de slide before compilation
  %Variable permettant d'activer ou non les pauses
  %This variable allow to activate or not pauses

  % 1 ==> Les pauses sont activées
  % 1 ==> Pauses are activated
  % 2 ==> Les pauses sont desactivées
  % 2 ==> Pauses aren't activated
  %\def\Pause{1}
  \def\Pause{2}


  %Variable permettant de choisir la langue (francais ou anglais)
  %This variable allow to chose the language (english or french)
  %\def\Language{french}
  \def\Language{english}
%fin de zone de configuration de compilation
%End of the configurate zone

\definecolor{colKeys}{rgb}{0,0,1} 
\definecolor{colIdentifier}{rgb}{0,0,0} 
\definecolor{colComments}{rgb}{0,0.5,1} 
\definecolor{colString}{rgb}{0.6,0.1,0.1} 

\lstset{%configuration de listings 
float=hbp,% 
basicstyle=\ttfamily\small, % 
identifierstyle=\color{colIdentifier}, % 
keywordstyle=\color{colKeys}, % 
stringstyle=\color{colString}, % 
commentstyle=\color{colComments}, % 
columns=flexible, % 
tabsize=2, % 
frame=trBL, % 
frameround=tttt, % 
extendedchars=true, % 
showspaces=false, % 
showstringspaces=false, % 
numbers=left, % 
numberstyle=\tiny, % 
breaklines=true, % 
breakautoindent=true, % 
captionpos=b,% 
xrightmargin=+0cm, % 
xleftmargin=+0cm
} 


\usetheme{Copenhagen}
\ifthenelse{\equal{\Language}{french}}
{
  \title[Introduction au C avr]{Programmation en C\\Pour avr 8 bits}
}
{
  \title[Introduction to C avr]{Programming C\\On 8 bits avr}
}
\author{J\'er\'emy Cheynet -- INTech \\ Yann Sionneau -- MiNET}
\institute{www.club-intech.fr \\ clubcode.minet.net \\ intlab.minet.net \\ github.com/leroilion/avr \\ www.slideshare.net/leroilion34}
\date{14 octobre 2010}
\logo{\includegraphics[height=0.5cm]{image/intech.png}}

\begin{document}
 
\begin{frame}
  \titlepage
\end{frame}

\begin{frame}
  \tableofcontents[]
\end{frame}
\ifthenelse{\equal{\Language}{french}}
{
\section{Les outils de programmation}
\subsection{Quels sont les outils dont je dispose ?}
}
{
\section{Programming tools}
\subsection{What are tools I can use ?}
}


\begin{frame}
  \tableofcontents[currentsection]
\end{frame}

\begin{frame}
\ifthenelse{\equal{\Language}{french}}
{
  \begin{block}{Les diff\'erents outils}
    \begin{itemize}
      \ifthenelse{\equal{\Pause}{1}}
      {
	\pause
      }
      {
      }
      \item avr-gcc (pour la compilation)
      \ifthenelse{\equal{\Pause}{1}}
      {
	\pause
      }
      {
      }
      \item avr-objcopy (pour cr\'eer le fichier hex)
      \ifthenelse{\equal{\Pause}{1}}
      {
	\pause
      }
      {
      }
      \item avrdude (pour flasher)
    \end{itemize}
  \end{block}
}
{
  \begin{block}{Differents tools}
    \begin{itemize}
      \ifthenelse{\equal{\Pause}{1}}
      {
	\pause
      }
      {
      }
      \item avr-gcc (to compile the code)
      \ifthenelse{\equal{\Pause}{1}}
      {
	\pause
      }
      {
      }
      \item avr-objcopy (to create the hexadecimal file)
      \ifthenelse{\equal{\Pause}{1}}
      {
	\pause
      }
      {
      }
      \item avrdude (to flash)
    \end{itemize}
  \end{block}
}
\end{frame}

\ifthenelse{\equal{\Language}{french}}
{
\subsection{Exemple d'utilisation}
}
{
\subsection{Use case}
}


\begin{frame}
  \begin{exampleblock}{avr-gcc}
    \ifthenelse{\equal{\Pause}{1}}
     {
	\pause
     }{}
    avr-gcc -Wall -mmcu=atmega328p -DF\_CPU=16000000 -c myfile1.c\\
    \ifthenelse{\equal{\Pause}{1}}
     {
	\pause
     }{}
    avr-gcc -Wall -mmcu=atmega328p -DF\_CPU=16000000 -c myfile2.c\\
    \ifthenelse{\equal{\Pause}{1}}
    {
	\pause
    }{}
    avr-gcc -Wall -mmcu=atmega328p -DF\_CPU=16000000 myfile1.o myfile2.o -o myappli 
  \end{exampleblock}
  \ifthenelse{\equal{\Pause}{1}}
  {
    \pause
  }{}
  \begin{exampleblock}{avr-objcopy}
    \ifthenelse{\equal{\Pause}{1}}
  {
    \pause
  }{}
    avr-objcopy -O ihex -R .eeprom myappli main.hex
  \end{exampleblock}
  \ifthenelse{\equal{\Pause}{1}}
  {
    \pause
  }{}
  \begin{exampleblock}{avrdude}
    \ifthenelse{\equal{\Pause}{1}}
  {
    \pause
  }{}
    sudo avrdude -P /dev/ttyUSB0 -c stk500v1 -p m328p -b 57600 -D -U flash:w:main.hex
  \end{exampleblock}
\end{frame}


\ifthenelse{\equal{\Language}{french}}
{
\subsection{A vous de jouer}
}
{
\subsection{Let's play now}
}

\begin{frame}
\ifthenelse{\equal{\Language}{french}}
{
\begin{block}{T\'el\'echarger les sources}
    \ifthenelse{\equal{\Pause}{1}}
  {
    \pause
  }{}
    \hyperlink{http://github.com/leroilion/avr}{http://github.com/leroilion/avr}\\
    Et t\'el\'echarger dans les exemples le code blink.c
  \end{block}
  \ifthenelse{\equal{\Pause}{1}}
  {
    \pause
  }{}
  \begin{block}{Compiler les sources}
    \ifthenelse{\equal{\Pause}{1}}
  {
    \pause
  }{}
    avr-gcc -Wall -mmcu=atmega328p -O2 -DF\_CPU=16000000 blink.c -o blink.out
  \end{block}
  \ifthenelse{\equal{\Pause}{1}}
  {
    \pause
  }{}
  \begin{block}{Cr\'eer le fichier hexad\'ecimal}
    \ifthenelse{\equal{\Pause}{1}}
  {
    \pause
  }{}
    avr-objcopy -O ihex -R .eeprom blink.out blink.hex
  \end{block}
  \ifthenelse{\equal{\Pause}{1}}
  {
    \pause
  }{}
  \begin{block}{Flasher l'arduino}
    \ifthenelse{\equal{\Pause}{1}}
  {
    \pause
  }{}
    sudo avrdude -P /dev/ttyUSB0 -c stk500v1 -p m328p -b 57600 -D -U flash:w:blink.hex\\
    sudo avrdude -c usbtiny -p m328p -U flash:w:blink.hex
  \end{block}
}
{
\begin{block}{Download source code}
    \ifthenelse{\equal{\Pause}{1}}
  {
    \pause
  }{}
    \hyperlink{http://github.com/leroilion/avr}{http://github.com/leroilion/avr}\\
    Download, in code example, the file blink.c
  \end{block}
  \ifthenelse{\equal{\Pause}{1}}
  {
    \pause
  }{}
  \begin{block}{Source compile}
    \ifthenelse{\equal{\Pause}{1}}
  {
    \pause
  }{}
    avr-gcc -Wall -mmcu=atmega328p -O2 -DF\_CPU=16000000 blink.c -o blink.out
  \end{block}
  \ifthenelse{\equal{\Pause}{1}}
  {
    \pause
  }{}
  \begin{block}{Create hexadecimal code}
    \ifthenelse{\equal{\Pause}{1}}
  {
    \pause
  }{}
    avr-objcopy -O ihex -R .eeprom blink.out blink.hex
  \end{block}
  \ifthenelse{\equal{\Pause}{1}}
  {
    \pause
  }{}
  \begin{block}{Flash arduino card}
    \ifthenelse{\equal{\Pause}{1}}
  {
    \pause
  }{}
    sudo avrdude -P /dev/ttyUSB0 -c stk500v1 -p m328p -b 57600 -D -U flash:w:blink.hex\\
    sudo avrdude -c usbtiny -p m328p -U flash:w:blink.hex
  \end{block}
}
\end{frame}

\section{Hello world !}


\begin{frame}
  \tableofcontents[currentsection]
\end{frame}

\ifthenelse{\equal{\Language}{french}}
{
\subsection{Structure d'un port}
}
{
\subsection{Port structure}
}

\ifthenelse{\equal{\Language}{french}}
{
\begin{frame}
\ifthenelse{\equal{\Pause}{1}}
{
  \pause
}{}
  \begin{block}{3 registres}
    \begin{itemize}
\ifthenelse{\equal{\Pause}{1}}
{
  \pause
}{}
      \item Le registre DDRx\\
      Registre de configuration du port
\ifthenelse{\equal{\Pause}{1}}
{
  \pause
}{}
      \item Le registre PORTx\\
      Registre de sortie du port
\ifthenelse{\equal{\Pause}{1}}
{
  \pause
}{}
      \item Le registre PINx\\
      Registre de lecture du port
    \end{itemize}
  \end{block}
\end{frame}
}
{
\begin{frame}
\ifthenelse{\equal{\Pause}{1}}
{
  \pause
}{}
  \begin{block}{3 registers}
    \begin{itemize}
\ifthenelse{\equal{\Pause}{1}}
{
  \pause
}{}
      \item The DDRx register\\
      This register allow to configure the type of the port (input ou output)
\ifthenelse{\equal{\Pause}{1}}
{
  \pause
}{}
      \item The PORTx register\\
      Output port register
\ifthenelse{\equal{\Pause}{1}}
{
  \pause
}{}
      \item The PINx register\\
      Input port register
    \end{itemize}
  \end{block}
\end{frame}
}

\ifthenelse{\equal{\Language}{french}}
{
\subsection{Ecrire un 1 ou un 0}
}
{
\subsection{Write 1 or 0}
}

\begin{frame}
\ifthenelse{\equal{\Pause}{1}}
{
  \pause
}{}
\ifthenelse{\equal{\Language}{french}}
{
  \begin{block}{Ecrire un 1 logique}
}
{
  \begin{block}{Write 1 logical}
}
\ifthenelse{\equal{\Pause}{1}}
{
  \pause
}{}
    myport $|$= ( 1 $<$$<$ mybit );\\
\ifthenelse{\equal{\Pause}{1}}
{
  \pause
}{}
    \#ifndef sbi\\
    \#define sbi(port,bit) (port) $|$= (1 $<$$<$ (bit))\\
    \#endif
  \end{block}
\ifthenelse{\equal{\Pause}{1}}
{
  \pause
}{}
\ifthenelse{\equal{\Language}{french}}
{
  \begin{block}{Ecrire un 0 logique}
}
{
  \begin{block}{Write 0 logical}
}
\ifthenelse{\equal{\Pause}{1}}
{
  \pause
}{}
    myport \&= $\sim$( 1 $<$$<$ mybit );\\
\ifthenelse{\equal{\Pause}{1}}
{
  \pause
}{}
    \#ifndef cbi\\
    \#define cbi(port,bit) (port) \&= $\sim$(1 $<$$<$ (bit))\\
    \#endif\\
  \end{block}
\end{frame}

\ifthenelse{\equal{\Language}{french}}
{
\subsection{Je te parle}
}
{
\subsection{I speak}
}

\begin{frame}
\ifthenelse{\equal{\Pause}{1}}
{
  \pause
}{}
  \begin{block}{Configuration en sortie du port}
\ifthenelse{\equal{\Pause}{1}}
{
  \pause
}{}
    DDRB $|$= ( 1 $<$$<$ PORTB5 );
  \end{block}
\ifthenelse{\equal{\Pause}{1}}
{
  \pause
}{}
  \begin{block}{Ecriture sur un port}
\ifthenelse{\equal{\Pause}{1}}
{
  \pause
}{}
    PORTB $|$= ( 1 $<$$<$ PORTB5 ); //Pour mettre le bit 5 du port B \`a 1\\
\ifthenelse{\equal{\Pause}{1}}
{
  \pause
}{}
    PORTB \&= $\sim$( 1 $<$$<$ PORTB5 ); //Pour mettre \`a 0 
  \end{block}
\end{frame}

\ifthenelse{\equal{\Language}{french}}
{
\subsection{Tu me parles}
}
{
\subsection{I listen}
}

\begin{frame}
\ifthenelse{\equal{\Pause}{1}}
{
  \pause
}{}
  \begin{block}{Configuration du port en entr\'ee}
\ifthenelse{\equal{\Pause}{1}}
{
  \pause
}{}
    DDRx \&= $\sim$( 1 $<$$<$ PORTxN );
  \end{block}
\ifthenelse{\equal{\Pause}{1}}
{
  \pause
}{}
  \begin{block}{Lecture du port}
\ifthenelse{\equal{\Pause}{1}}
{
  \pause
}{}
    PORTx $|$= ( 1 $<$$<$ PORTxN ); //Pour activer le pull--up\\
    PORTx \&= $\sim$( 1 $<$$<$ PORTxN ); //Pour d\'esactiver le pull--up\\
\ifthenelse{\equal{\Pause}{1}}
{
  \pause
}{}
    uint8\_t etat = ( PINx \& ( 1 $<$$<$ PINxN ));
  \end{block}
\end{frame}

\subsection{A vous de jouer}

\begin{frame}
  \underline{Objectif:}\\
  Faire un programme qui \'eclaire une LED si un bouton est pouss\'e.\\
\ifthenelse{\equal{\Pause}{1}}
{
  \pause
}{}
  \underline{D\'etails techniques:}\\
  Utiliser le PORTB5 en sortie (PORTB7 pour arduino mega), et le PORTB0 en entr\'ee.
\ifthenelse{\equal{\Pause}{1}}
{
  \pause
}{}
  \begin{alertblock}{Attention}
    Ne pas oublier le int main() dans le fichier principal.\\
    Penser \`a rajouter l'include standard io.h
  \end{alertblock}
\end{frame}

\lstset{language=c++} 
\lstset{commentstyle=\textit} 
\begin{lstlisting}
#include <avr/io.h>
int main( void )
{
        DDRB |= ( 1 << PORTB5 );
        DDRB &= ~( 1 << PORTB0 );
        PORTB |= ( 1 << PORTB0 );
        while(42)
        {
                if( (PINB & ( 1 << PORTB0 )) )
                        PORTB |= ( 1 << PORTB5 );
                else
                        PORTB &= ~( 1 << PORTB5 );
        }
        return 0;
}
\end{lstlisting} 

\section{C'est bien, mais comment je peux faire un VRAI programme ?}

\begin{frame}
  \tableofcontents[currentsection]
\end{frame}

\subsection{Qu'est-ce qu'un registe ?}

\begin{frame}
  \begin{block}{}
    C'est un octet en m\'emoire\\
\ifthenelse{\equal{\Pause}{1}}
{
  \pause
}{}
    qui permet de configurer le microcontr\^oleur.
\ifthenelse{\equal{\Pause}{1}}
{
  \pause
}{}
  \end{block}
\ifthenelse{\equal{\Pause}{1}}
{
  \pause
}{}
  \begin{exampleblock}{SREG -- AVR Status Register}
    \begin{itemize}
\ifthenelse{\equal{\Pause}{1}}
{
  \pause
}{}
      \item I (bit 7) -- Global interrupt enable
\ifthenelse{\equal{\Pause}{1}}
{
  \pause
}{}
      \item T (bit 6) -- Copy storage
\ifthenelse{\equal{\Pause}{1}}
{
  \pause
}{}
      \item H (bit 5) -- Half carry
\ifthenelse{\equal{\Pause}{1}}
{
  \pause
}{}
      \item S (bit 4) -- Sign bit
\ifthenelse{\equal{\Pause}{1}}
{
  \pause
}{}
      \item V (bit 3) -- Overflow bit
\ifthenelse{\equal{\Pause}{1}}
{
  \pause
}{}
      \item N (bit 2) -- Negative bit
\ifthenelse{\equal{\Pause}{1}}
{
  \pause
}{}
      \item Z (bit 1) -- Zero bit
\ifthenelse{\equal{\Pause}{1}}
{
  \pause
}{}
      \item C (bit 0) -- Carry
    \end{itemize}
  \end{exampleblock}
\end{frame}

\subsection{Le fil rouge}

\begin{frame}
  \underline{Objectif:}\\
  Faire un programme qui fait clignoter une led en utilisant le TIMER1 sur 16 bits.\\Pour cela, on fera:\\
  \begin{itemize}
\ifthenelse{\equal{\Pause}{1}}
{
  \pause
}{}
    \item On activera les interruptions d'overflow du TIMER1
\ifthenelse{\equal{\Pause}{1}}
{
  \pause
}{}
    \item On fera compter le TIMER1 pour avoir une interruption toutes les secondes.
\ifthenelse{\equal{\Pause}{1}}
{
  \pause
}{}
    \item On regardera l'\'etat de la PIN associ\'ee \`a la LED pour le changer.
  \end{itemize}
\end{frame}


\subsection{Le datasheet: la bible du programmateur bas niveau}

\begin{frame}
  \hyperlink{http://github.com/leroilion/avr/tree/master/datasheet/Atmegaxx8.pdf}{http://github.com/leroilion/avr}
\ifthenelse{\equal{\Pause}{1}}
{
  \pause
}{}
  \begin{block}{TCCR1A -- TCCR1B (page 132 -- 134)}
    Mode normal (WGMx = 0 ), Source d'horloge (CSx = 101), Pas de comparaison (COMx = 0) 
  \end{block}
\ifthenelse{\equal{\Pause}{1}}
{
  \pause
}{}
  \begin{block}{TCNT1H -- TCNT1L (page 136) }
    Incr\'ementation toute les 64$\mu$s \pause $\Longrightarrow$ Compter jusqu'\`a 15625 \pause $\Longrightarrow$ Mettre 49910 = 65535 - 15625 dans TCNT.
  \end{block}
\ifthenelse{\equal{\Pause}{1}}
{
  \pause
}{}
  \begin{block}{TIMSK1 (page 137)}
    Activer l'interruption d'overflow \pause $\Longrightarrow$ Activer TOIE1
  \end{block}
\end{frame}

\subsection{Tu me vois, tu me vois plus}

\lstset{language=c++} 
\lstset{commentstyle=\textit} 
\begin{lstlisting}
#include <avr/io.h>
#include <avr/interrupt.h>
...
TCCR1A = 0bxxxxxx00;
TCCR1B = 0bxxx00101;
TIMSK1 |= ( 1 << TOIE1 );
...
ISR( TIMER1_OVF_vect)
{ ...
\end{lstlisting}

\subsection{A vous de jouer}

\begin{frame}
  \underline{Objectif:}\\
  Faire un programme qui permet de contr\^oler la luminosit\'e d'une LED en utilisant le PWM.\\
\ifthenelse{\equal{\Pause}{1}}
{
  \pause
}{}
  \underline{D\'etails techniques:}\\
  \begin{itemize}
\ifthenelse{\equal{\Pause}{1}}
{
  \pause
}{}
    \item Utiliser le port B5 pour la sortie de la LED (PORTB7 pour arduino mega).
\ifthenelse{\equal{\Pause}{1}}
{
  \pause
}{}
    \item Utiliser le TIMER de votre choix en mode PWM
  \end{itemize}

\ifthenelse{\equal{\Pause}{1}}
{
  \pause
}{}
  \begin{alertblock}{Attention}
    Ne pas oublier l'include $<$avr/interrupt.h$>$\\
    Penser \`a g\'erer \emph{TOUS} les vecteurs d'interruption
  \end{alertblock}
\end{frame}

\section{Les pi\`eges \`a \'eviter}

\begin{frame}
  \begin{alertblock}{Les pi\`eges}
\ifthenelse{\equal{\Pause}{1}}
{
  \pause
}{}
    \begin{itemize}
     \item \'Economiser la m\'emoire (probl\`eme de la pile)
\ifthenelse{\equal{\Pause}{1}}
{
  \pause
}{}
     \item Faire attention avec les \emph{float}, les . et les \emph{double}
\ifthenelse{\equal{\Pause}{1}}
{
  \pause
}{}
     \item \'Economiser la puissance de calcul (calcul en 8 bits)
\ifthenelse{\equal{\Pause}{1}}
{
  \pause
}{}
     \item Faire attention \`a l'overflow
\ifthenelse{\equal{\Pause}{1}}
{
  \pause
}{}
     \item Rajouter l'option \emph{volatile} devant les variables
    \end{itemize}

  \end{alertblock}
\end{frame}


\section*{Conclusion}

\begin{frame}
  Nous avons vu :\\
  \begin{itemize}
\ifthenelse{\equal{\Pause}{1}}
{
  \pause
}{}
    \item Les outils de programmation
\ifthenelse{\equal{\Pause}{1}}
{
  \pause
}{}
\ifthenelse{\equal{\Pause}{1}}
{
  \pause
}{}
    \item Ce qu'est un registre, et comment le configurer \`a l'aide du datasheet
\ifthenelse{\equal{\Pause}{1}}
{
  \pause
}{}
    \item L'utilisation des interruptions
  \end{itemize}
\ifthenelse{\equal{\Pause}{1}}
{
  \pause
}{}
  Des exemples simple :\\
   \begin{itemize}
\ifthenelse{\equal{\Pause}{1}}
{
  \pause
}{}
    \item \hyperlink{http://github.com/leroilion/avr}{http://github.com/leroilion/avr}
   \end{itemize}
\ifthenelse{\equal{\Pause}{1}}
{
  \pause
}{}
  Bibliographie :\\
    \begin{itemize}
\ifthenelse{\equal{\Pause}{1}}
{
  \pause
}{}
      \item \hyperlink{http://www.tavernier-c.com/livres.htm}{\underline{Microcontr\^oleurs AVR : des ATtiny aux ATmega} de Christian Tavernier }
    \end{itemize}

\end{frame}

\end{document}
